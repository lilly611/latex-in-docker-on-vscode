\documentclass[a4j,uplatex]{jsarticle}
\usepackage{amsmath}
\usepackage{amsthm}
\usepackage{amssymb}
\usepackage{amsfonts}
\usepackage{physics}

\theoremstyle{definition}
\newtheorem{dfn}{Definition}[section]
\newtheorem*{df*}{Definition}
\newtheorem{prop}[dfn]{Proposition}
\newtheorem{lem}[dfn]{Lemma}
\newtheorem{thm}[dfn]{Theorem}
\newtheorem{cor}[dfn]{Corollary}
\newtheorem{rem}[dfn]{Remark}
\newtheorem{fact}[dfn]{Fact}

\title{数理最適化大意 課題1}
\author{理学部4年 照屋佑喜仁}

\begin{document}
\maketitle
\section{問題1}
$b\in \mathbb{R}^m,\, c \in \mathbb{R}^n,\, A\in\mathbb{R}^{m\times n}$を定数ベクトル,定数行列とする.\\
次の線形計画問題を標準形に書き換える.
\begin{align*}
    \inf \left\{b^T y: c-A^Ty \geq  0 ,\,y \in \mathbb{R}^m\right\}\tag{1}
\end{align*}
まず,$s=c-A^Ty$と置くことで
\begin{align*}
    (1) \; = \inf\left\{ b^Ty:c-A^Ty=s,s\geq 0,y\in\mathbb{R}^m,s\in\mathbb{R}^n \right\}
\end{align*}
ここで,$y=y^+-y^-,\; y^+,y^-\geq 0$と置けば
\begin{align*}
    (1) & =\inf \left\{ b^T(y^+-y^-):\, c-A^T(y^+-y^-)=s,\: y^+,y^-,s\geq 0,\: s\in\mathbb{R}^n,\: y^+,y^-\in\mathbb{R}^m \right\} \\
        & =\inf \left\{ \left(\begin{matrix}
                                      b \\-b\\0
                                  \end{matrix}\right)^T
    \left(
    \begin{matrix}
            y^+ \\y^-\\s
        \end{matrix}
    \right)
    :(A^T,-A^T,I_n)
    \left(
    \begin{matrix}
            y^+ \\y^-\\s
        \end{matrix}
    \right)
    =c,\,
    \left(
    \begin{matrix}
            y^+ \\y^-\\s
        \end{matrix}
    \right)\geq 0, \,     \left(
    \begin{matrix}
            y^+ \\y^-\\s
        \end{matrix}
    \right)\in \mathbb{R}^{2m+n}
    \right\}
\end{align*}
$\left(\begin{matrix}
            b \\-b\\0
        \end{matrix}\right)^T=\tilde{b},\:     \left(
    \begin{matrix}
            y^+ \\y^-\\s
        \end{matrix}
    \right)=\tilde{x},\,(A^T,-A^T,I_n)=\tilde{A}$とすることで

\begin{align*}
    (1)=\inf \left\{ \tilde{b}^T\tilde{x}:\; \tilde{A}\tilde{x}=c,\; \tilde{x}\geq 0,\; \tilde{x}\in\mathbb{R}^2m+n\right\}
\end{align*}
と標準形で書くことができた.$\square$

\section{問題2}
\subsection{Q1}
次の線形計画問題の双対問題について考える.
\begin{align*}
    \inf \left\{ x:\; x+y+z=72,\, x+y\geq 50,\,x+z\geq 32,\,y+z\geq 20,\; x,y,z\geq 0 \right\}\tag{2}
\end{align*}
まず,この線形計画問題は次と同値である.
\begin{align*}
    \inf \left\{x:\; x+y+z=72,\,x+y-50=a,x+z-32=b,y+z-20=c,\; x,y,z,a,b,c\geq 0\right\}
\end{align*}
ここで
$A=
    \begin{pmatrix}
        1 & 1 & 1 & 0  & 0  & 0  \\
        1 & 1 & 0 & -1 & 0  & 0  \\
        1 & 0 & 1 & 0  & -1 & 0  \\
        0 & 1 & 1 & 0  & 0  & -1
    \end{pmatrix},\boldsymbol{x}=\begin{pmatrix}
        x & y & z & a & b & c
    \end{pmatrix}^T,
    \boldsymbol{b}=\begin{pmatrix}
        72 & 50 & 32 & 20
    \end{pmatrix}^T,\\
    \boldsymbol{c}=\begin{pmatrix}
        1 & 0 & 0 & 0 & 0 & 0
    \end{pmatrix}^T
$と置けば
\begin{align*}
    \inf \left\{\boldsymbol{c}^T\boldsymbol{x}:\; A\boldsymbol{x}=\boldsymbol{b},\: \boldsymbol{x}\geq 0,\: \boldsymbol{x}\in\mathbb{R}^6\right\}
\end{align*}
とも同値である.\\
双対問題を考えるにあたり,もとの線形計画問題を下から評価する$\boldsymbol{b}^T\boldsymbol{y}$を考え,それを最大化させると考える.\\
$\boldsymbol{y}\in \mathbb{R}^6$とする.
\begin{align*}
    \boldsymbol{b}^T\boldsymbol{y}=(A\boldsymbol{x})^T\boldsymbol{y} & =\boldsymbol{x}^TA^T\boldsymbol{y}                                                             \\
                                                                     & =\left\{(A^T\boldsymbol{y})^T\boldsymbol{x}\right\}^T                                          \\
                                                                     & =(A^T\boldsymbol{y})^T\boldsymbol{x} \quad \because \boldsymbol{b}^T\boldsymbol{y}\text{はスカラー}
\end{align*}
$\boldsymbol{x}\geq0$を満たすので,係数ベクトル$\boldsymbol{y}$が$A^T\boldsymbol{y}\geq \boldsymbol{c}$を満たせば
\begin{align*}
    \boldsymbol{b}^T\boldsymbol{y}=(A^T\boldsymbol{y})^T\boldsymbol{x}\leq \boldsymbol{c}^T\boldsymbol{x}
\end{align*}
となり,もとの線形計画問題の目的関数値を下から評価できる.よって求める双対問題は
\begin{align*}
     & \sup \left\{\boldsymbol{b}^T\boldsymbol{y}:\; A^T\boldsymbol{y}\geq \boldsymbol{c},\; \boldsymbol{y}\in\mathbb{R}^4\right\}                                                                              \\
     & =\sup \left\{\boldsymbol{b}^T\boldsymbol{y}:\; \boldsymbol{c}-A^T\boldsymbol{y}=\boldsymbol{z} ,\; \boldsymbol{z}\geq 0,\; \boldsymbol{y}\in\mathbb{R}^4,\; \boldsymbol{z}\in\mathbb{R}^6\right\}\tag{3}
\end{align*}$\square$

\subsection{Q2}
まず,(2)の実行可能解で目的関数$x$をできるだけ小さくするようなものを考えてみる.条件より
\begin{align*}
    72-z\geq50\quad \therefore z\leq22 \\
    72-y\geq32\quad \therefore y\leq40 \\
    72-x\geq20\quad \therefore x\leq52
\end{align*}
が直ちにわかる.\\
ここで$x+y\geq50$に注目すると$x$が最小となるのは$y$ができるだけ大きいときと考察できる.上の条件から,$y=40$とすると,$x\geq10$となり,その最小値$x=10$を取れば$x+y+z=72$より$z=22$となりこれはどの条件も満たすため$(x,y,z)=(10,40,22)$は実行可能解である.この実行可能解を(2)の表記に合わせて$\boldsymbol{x}^*=(10,40,22,0,0,42)$とおく.\\
(3)の実行可能解を$\boldsymbol{y}^*,\boldsymbol{z}^*$としておく.仮に$\boldsymbol{x}^*$が最小解であるとして,相補性定理より$\boldsymbol{x}^*_i\boldsymbol{z}^*_i=0,\; (i=1,2,\cdots,6)$を解く.
$\boldsymbol{x}^*_i> 0$のとき$\boldsymbol{z}^*_i=0$つまり$(A^T\boldsymbol{y}^*)_i=\boldsymbol{c}_i$を解けば良いので,($\boldsymbol{x}_i=0$のとき$\boldsymbol{z}^*_i$は任意の値で相補性を満たすため考えなくてよい.)
\begin{align}
    \begin{cases}
        y_1+y_2+y_3=1 \\
        y_1+y_2+y_4=0 \\
        y_1+y_3+y_4=0 \\
        -y_4=0
    \end{cases}
\end{align}
これを解いて$\boldsymbol{y}^*=(-1,1,1,0)^T$これは(3)の条件を満たすため実行可能解.(ちなみに$\boldsymbol{z}^*=(0,0,0,0,1,0)^T$となった.確かに条件を満たす.)\\
よってこれは最適解である.目的関数の値は$\boldsymbol{b}^T\boldsymbol{y}^*=10$となり最小値に一致する.最小値と最適値は一致し$10$である.
\end{document}
