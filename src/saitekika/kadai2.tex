\documentclass[a4j,uplatex]{jsarticle}
\usepackage{amsmath}
\usepackage{amsthm}
\usepackage{amssymb}
\usepackage{amsfonts}
\usepackage{physics}

\theoremstyle{definition}
\newtheorem{dfn}{Definition}[section]
\newtheorem*{df*}{Definition}
\newtheorem{prop}[dfn]{Proposition}
\newtheorem{lem}[dfn]{Lemma}
\newtheorem{thm}[dfn]{Theorem}
\newtheorem{cor}[dfn]{Corollary}
\newtheorem{rem}[dfn]{Remark}
\newtheorem{fact}[dfn]{Fact}
%定義の:=を書くためのやつ
\def\coloneqq{\mathrel{\mathop:}=}%

\title{数理最適化大意 課題2}
\author{理学部4年 照屋佑喜仁}

\begin{document}
\maketitle
\section{問題1}
\subsection{P1}
まずLKPを標準形で表すことを考える.LKPは次のように表すことができる.
\begin{align*}
    p^* & \coloneqq \max \left\{\sum_{i=1}^{n}v_ix_i:\sum_{i=1}^{n}w_ix_i\leq W,\; 0\leq x\leq 1,\;(i=1,\dots,n)\right\}         \\
        & =\max\left\{\sum_{i=1}^{n}v_ix_i:\sum_{i=1}^{n}w_ix_i+q=W,\;x_i+p_i=1,x_i\geq0,q\geq0,p_i\geq0,\;(i=1,\dots,n)\right\}
\end{align*}
ここで,
\begin{align*}
    A         & =\begin{pmatrix}
                     w_1    & w_2 & \dots  & w_n & 0      & 0 & 0      & \dots & 0 & 1      \\
                     1      & 0   & \dots  & 0   & 1      & 0 & 0      & \dots & 0 & 0      \\
                     0      & 1   & \dots  & 0   & 0      & 1 & 0      & \dots & 0 & 0      \\
                     \vdots &     & \ddots &     & \vdots &   & \ddots &       &   & \vdots \\
                     0      & 0   & \dots  & 1   & 0      & 0 & 0      & \dots & 1 & 0
                 \end{pmatrix} \\
    \tilde{x} & =(x_1,\dots,x_n,p_1,\dots,p_n,q)^T                                      \\
    b         & =(W,1,\dots,1)^T                                                        \\
    c         & =(v_1,\dots,v_n,0,\dots,0)
\end{align*}
と置くことにより
\begin{align*}
    p^* & =\sup\left\{c^T\tilde{x}:A\tilde{x}=b,\;\tilde{x}\geq0\right\}
\end{align*}
標準形にするために$\inf$で表したい.$\tilde{c}\coloneqq -c$とおいて
\begin{align*}
    p^*=\inf\left\{\tilde{c}^T\tilde{x}:A\tilde{x}=b,\;\tilde{x}\geq0\right\}\tag{1}
\end{align*}
これがKLPの標準形である.
よって双対問題は
\begin{align*}
    d^*\coloneqq\sup\left\{b^Ty:A^Ty\leq \tilde{c}\right\}\tag{2}
\end{align*}
となり,
\begin{align*}
    y_i=\begin{cases}
            -t\quad i=1 \\
            -s_{i-1} \quad i \geq 2
        \end{cases}
\end{align*}
と置いて目的関数を$-1$倍すれば
\begin{align*}
    d^*=\min\left\{Wt+\sum_{i=1}^{n}s_i:w_it+s_i\geq v_i\;(i=1,\dots,n),t,s_1,\dots,s_n\geq0\right\}
\end{align*}
を得る.(証明終)
\subsection{P2}
まず,与えられた解が(1),(2)の実行解か確かめて,相補性定理を用いて最適解かどうか確かめる.与えられた解から$\tilde{x}$を求めることができて($x_i+p_i=1$などの条件を用いるとわかる)
\begin{align*}
    \tilde{x}_i=\begin{cases}
                    1\quad                                   & i=\begin{cases}
                                                         1,\dots,k \\n+k+2,\dots,2n
                                                     \end{cases} \\
                    \frac{W-\sum_{i=1}^{k}w_i}{w_{k+1}}\quad & i=k+1                       \\
                    0\quad                                   & i=\begin{cases}
                                                         k+2,\dots,n+k \\2n+1
                                                     \end{cases}       \\
                    \frac{\sum_{i=1}^{k+1}w_i-W}{w_{k+1}}    & i=n+k+1\quad                \\
                \end{cases}
\end{align*}
となる.$w_i$についての仮定より$\tilde{x}\geq0$なので実行可能解である.\vskip.5\baselineskip
また,与えられた解から双対問題の解は
\begin{align*}
    y_i=\begin{cases}
            -\frac{v_{k+1}}{w_{k+1}}\quad               & i=1             \\
            w_{i-1}\frac{v_{k+1}}{w_{k+1}}-v_{i-1}\quad & i=2,\dots,k+1   \\
            0\quad                                      & i=k+2,\dots,n+1
        \end{cases}
\end{align*}
となり,$v_i,w_i$の比の仮定からこれは実行可能解であった.($A^Ty\leq c$を満たす.)
ここで,
\begin{align*}
    c-A^Ty & =\begin{pmatrix}
                  -v_1-\left\{w_1(-\frac{v_k+1}{w_{k+1}})+w_1\frac{v_{k+1}}{w_{k+1}}-v_1\right\} \\
                  -v_2-\left\{w_2(-\frac{v_k+1}{w_{k+1}})+w_2\frac{v_{k+1}}{w_{k+1}}-v_2\right\} \\
                  \vdots                                                                         \\
                  -v_k-\left\{w_k(-\frac{v_k+1}{w_{k+1}})+w_k\frac{v_{k+1}}{w_{k+1}}-v_k\right\} \\
                  -v_{k+1}-\left\{w_{k+1}(-\frac{v_{k+1}}{w_{k+1}})\right\}                      \\
                  \vdots                                                                         \\
                  -v_n-\left\{w_n(-\frac{v_{k+1}}{w_{k+1}})\right\}                              \\
                  -\left\{w_1\frac{v_{k+1}}{w_{k+1}}-v_1\right\}                                 \\
                  -\left\{w_k\frac{v_{k+1}}{w_{k+1}}-v_k\right\}                                 \\
                  0                                                                              \\
                  \vdots                                                                         \\
                  0                                                                              \\
                  -\left(-\frac{v_{k+1}}{w_{k+1}}\right)
              \end{pmatrix} \\
\end{align*}
\begin{align*}
    \phantom{c-A^Ty} & = \begin{pmatrix}
                             0                                                         \\
                             \vdots                                                    \\
                             0                                                         \\
                             0                                                         \\
                             -v_{k+2}-\left\{w_{k+2}(-\frac{v_{k+1}}{w_{k+1}})\right\} \\
                             \vdots                                                    \\
                             -v_n-\left\{w_n(-\frac{v_{k+1}}{w_{k+1}})\right\}         \\
                             -\left\{w_1\frac{v_{k+1}}{w_{k+1}}-v_1\right\}            \\
                             -\left\{w_k\frac{v_{k+1}}{w_{k+1}}-v_k\right\}            \\
                             0                                                         \\
                             \vdots                                                    \\
                             0                                                         \\
                             -\left(-\frac{v_{k+1}}{w_{k+1}}\right)
                         \end{pmatrix}
\end{align*}
よって$\tilde{x}_i\left(c-A^Ty\right)_i$について
\begin{align*}
    \begin{cases}
        i=1,\dots,k+1\text{のとき}\quad \left(c-A^T\right)_i=0    \\
        i=k+2,\dots,n+k\text{のとき}\quad \tilde{x}_i=0           \\
        i=n+k+1,\dots,2n\text{のとき}\quad \left(c-A^T\right)_i=0 \\
        i=2n+1\text{のとき}\quad \tilde{x}_i=0
    \end{cases}
\end{align*}
よって$\tilde{x}_i\left(c-A^Ty\right)_i=0$が成り立つから,与えられた解は最適解でありそれぞれLKP,DKPの最大解,最小解であることが示された.
\section{問題2}
\begin{itemize}
    \item{$B_1=\left\{1,2,4,5\right\}$} \\添字集合$B_1$から次の連立方程式
    \begin{align*}
        \begin{cases}
            x_1+x_2=72     \\
            x_1+x_2-x_4=50 \\
            x_1-x_5=32     \\
            x_2=20
        \end{cases}
    \end{align*}
    を得る.これを解くと$x_1=52,x_2=20,x_4=22,x_5=20$を得る.したがって,$(52,20,0,22,20,0)$は基底解であり非負なので実行可能基底解である.
    \item{$B_2=\left\{1,2,3,6\right\}$} \\添字集合$B_1$から次の連立方程式
    \begin{align*}
        \begin{cases}
            x_1+x_2+x_3=72 \\
            x_1+x_2=50     \\
            x_1+x_3=32     \\
            x_2+x_3=20
        \end{cases}
    \end{align*}
    を得る.これを解くと$x_1=10,x_2=40,x_3=22,x_6=42$を得る.したがって,$(10,40,22,0,0,42)$は基底解であり非負なので実行可能基底解である.
\end{itemize}
\end{document}
