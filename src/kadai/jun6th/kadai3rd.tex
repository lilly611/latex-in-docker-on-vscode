\documentclass[a4j,uplatex]{jsarticle}
\usepackage{amsmath}
\usepackage{amsthm}
\usepackage{amssymb}
\usepackage{amsfonts}
\usepackage{physics}
\usepackage{geometry}
\geometry{margin=2.5cm}
\usepackage{xcolor}
\newcommand{\speaker}[1]{\textbf{\textcolor{blue}{#1}}}
\newcommand{\speakerA}[1]{\textbf{\textcolor{red}{#1}}}
\newcommand{\speakerB}[1]{\textbf{\textcolor{green!60!black}{#1}}}
\newenvironment{dialogue}
{%
    \begin{quote}
        \setlength{\parskip}{0.5em}
        \setlength{\parindent}{0em}
        }
        {%
    \end{quote}
}

\newcommand{\say}[2]{\speaker{#1:} #2\par}
\newcommand{\sayA}[2]{\speakerA{#1:} #2\par}
\newcommand{\sayB}[2]{\speakerB{#1:} #2\par}

\theoremstyle{definition}
\newtheorem{dfn}{Definition}[section]
\newtheorem*{df*}{Definition}
\newtheorem{prop}[dfn]{Proposition}
\newtheorem{lem}[dfn]{Lemma}
\newtheorem{thm}[dfn]{Theorem}
\newtheorem{cor}[dfn]{Corollary}
\newtheorem{rem}[dfn]{Remark}
\newtheorem{fact}[dfn]{Fact}
\newtheorem*{problem*}{問題}
%定義の:=を書くためのやつ
\def\coloneqq{\mathrel{\mathop:}=}%
\DeclareMathOperator{\loss}{Loss}

\title{数理・人工知能のフロンティアと社会価値創造 課題 3日目}
\author{理学部4年 1SC22317Y 照屋佑喜仁}

\begin{document}
\maketitle
Q1について日本語で述べる.\vskip.5\baselineskip
私の専門分野は統計、特にベイズ統計である。これは、データから確率的な推論を行い、不確実性を定量的に扱う手法である。ABM(エージェントベースモデリング)は、個々のエージェントの行動と相互作用から社会システムを再現するシミュレーション手法であり、この両者を統合することで、社会問題解決に貢献できると考える。
エージェントの意思決定確率や学習率といったパラメータは、現実世界のデータから統計的に推定し、その不確実性を含めてモデルに組み込める。これにより、主観的な仮定に頼らず、データに基づいたより現実的なエージェント行動をシミュレートできる。
さらに、ABMの出力するマクロな現象と現実の観測データを比較する際、ベイズ的な手法を用いて、最適なエージェントパラメータを推定できると考えられる。例えば、新しい公共サービス導入時の利用者の行動変容予測において、統計モデルで個人の属性と選択行動の関係を推定し、ABMに適用することで、より正確な政策効果を評価できる。
このように、統計、特にベイズ統計はABMにデータ駆動型の厳密な推論と不確実性評価の枠組みをもたらす。これはエージェント行動モデリングの精度向上とシミュレーション結果の信頼性向上に直結し、複雑な社会問題を深く理解し、データに基づいた効果的な政策立案を支援する強力なツールとなるだろう。(572 words)

\end{document}
