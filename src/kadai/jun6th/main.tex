\documentclass[a4j,uplatex]{jsarticle}
\usepackage{amsmath}
\usepackage{amsthm}
\usepackage{amssymb}
\usepackage{amsfonts}
\usepackage{physics}
\usepackage{geometry}
\geometry{margin=2.5cm}
\usepackage{xcolor}
\newcommand{\speaker}[1]{\textbf{\textcolor{blue}{#1}}}
\newcommand{\speakerA}[1]{\textbf{\textcolor{red}{#1}}}
\newcommand{\speakerB}[1]{\textbf{\textcolor{green!60!black}{#1}}}
\newenvironment{dialogue}
{%
    \begin{quote}
        \setlength{\parskip}{0.5em}
        \setlength{\parindent}{0em}
        }
        {%
    \end{quote}
}

\newcommand{\say}[2]{\speaker{#1:} #2\par}
\newcommand{\sayA}[2]{\speakerA{#1:} #2\par}
\newcommand{\sayB}[2]{\speakerB{#1:} #2\par}

\theoremstyle{definition}
\newtheorem{dfn}{Definition}[section]
\newtheorem*{df*}{Definition}
\newtheorem{prop}[dfn]{Proposition}
\newtheorem{lem}[dfn]{Lemma}
\newtheorem{thm}[dfn]{Theorem}
\newtheorem{cor}[dfn]{Corollary}
\newtheorem{rem}[dfn]{Remark}
\newtheorem{fact}[dfn]{Fact}
\newtheorem*{problem*}{問題}
%定義の:=を書くためのやつ
\def\coloneqq{\mathrel{\mathop:}=}%

\title{数理・人工知能のフロンティアと社会価値創造 課題 1日目}
\author{理学部4年 1SC22317Y 照屋佑喜仁}

\begin{document}
\maketitle
\section{大学入試数学をchatGPT-4oに解かせる}
2025年度における九州大学の大学入試数学大問5の問題をchatGPT-4oモデルに回答させた.
大問5は、確率と代数の複合問題だが実際に重要なのは代数の知識であり,判別式や解の条件などを考える必要がある。
以下に問題を示す。

\begin{problem*}
    1個のさいころを3回続けて投げ,出る目を順に$a,b,c$とする.整式
    \begin{align*}
        f(x)=(x^2-ax+b)(x-c)
    \end{align*}
    について,以下の問いに答えよ.
    \begin{itemize}
        \item $f(x)=0$をみたす実数$x$の個数が1個である確率を求めよ。
        \item $f(x)=0$をみたす実数$x$の個数が3個である確率を求めよ。
    \end{itemize}
\end{problem*}

初め,熟考のオプションは指定せずに解かせたところ2つの問題のうち前者は間違え後者は正しく回答した.以下にそのやり取りを示す.\vskip.5\baselineskip

\begin{dialogue}
    \sayA{自分}{(画像を添付) 次の画像は大学入試に使われる数学の問題です。回答してください。その際,解答ではなく、実際に受験生が提出する体で回答を作成してください。}
    \sayB{chatGPT}{1個のさいころを3回投げ、出た目を順に $a,b,c$ とする。このとき、与えられた関数
        $$f(x) = (x^2 - ax + b)(x - c)$$
        について以下の問いに答える。

        \textbf{解答}

        \textbf{(1) $f(x) = 0$ を満たす実数解の個数が1個である確率}

        $f(x) = (x^2 - ax + b)(x - c) = 0$ の解は、$x^2 - ax + b = 0$ の解と $x = c$ である。

        実数解が1個となるのは、$x^2 - ax + b = 0$ が実数解を持たず、$x = c$ のみが実数解となる場合である。

        $x^2 - ax + b = 0$ が実数解を持たない条件は、判別式 $D = a^2 - 4b < 0$ である。

        $a, b, c$ はさいころの出目なので $1 \leq a, b, c \leq 6$ であり、全体の場合の数は $6^3 = 216$ 通りである。

        $a^2 - 4b < 0$ すなわち $b > \frac{a^2}{4}$ となる $(a, b)$ の組み合わせを数える。

        $a = 1$:$b > \frac{1}{4}$ より $b = 1, 2, 3, 4, 5, 6$ で 6通り

        $a = 2$:$b > 1$ より $b = 2, 3, 4, 5, 6$ で 5通り

        $a = 3$:$b > \frac{9}{4} = 2.25$ より $b = 3, 4, 5, 6$ で 4通り

        $a = 4$:$b > 4$ より $b = 5, 6$ で 2通り

        $a = 5$:$b > \frac{25}{4} = 6.25$ より該当なしで 0通り

        $a = 6$:$b > 9$ より該当なしで 0通り

        よって $(a, b)$ の組み合わせは $6 + 5 + 4 + 2 + 0 + 0 = 17$ 通り。

        $c$ は任意の値(1〜6)を取れるので、条件を満たす $(a, b, c)$ の組み合わせは $17 \times 6 = 102$ 通り。

        したがって求める確率は $\frac{102}{216} = \frac{17}{36}$

        \textbf{(2) $f(x) = 0$ を満たす実数解の個数が3個である確率}

        $f(x) = 0$ が3個の実数解を持つには、$x^2 - ax + b = 0$ が2個の異なる実数解を持ち、かつその2解が $c$ と異なる必要がある。

        $x^2 - ax + b = 0$ が異なる2実解を持つ条件は $a^2 - 4b > 0$ である。

        $x^2 - ax + b = 0$ の2解を $\alpha, \beta$ とすると、解と係数の関係より
        $$\alpha + \beta = a, \quad \alpha \beta = b$$

        $\alpha, \beta$ が異なる自然数で、かつ $c$ も $\alpha, \beta$ と異なる自然数である場合を考える。

        さいころの出目は1〜6なので、$1 \leq \alpha < \beta \leq 6$ かつ $1 \leq c \leq 6$ かつ $c \neq \alpha, \beta$ の場合を数える。

        可能な $(\alpha, \beta)$ の組み合わせと対応する $(a, b)$:

        $(\alpha, \beta) = (1, 2)$:$(a, b) = (3, 2)$
        $(\alpha, \beta) = (1, 3)$:$(a, b) = (4, 3)$
        $(\alpha, \beta) = (1, 4)$:$(a, b) = (5, 4)$
        $(\alpha, \beta) = (1, 5)$:$(a, b) = (6, 5)$
        $(\alpha, \beta) = (1, 6)$:$(a, b) = (7, 6)$ → $a = 7$ は不可
        $(\alpha, \beta) = (2, 3)$:$(a, b) = (5, 6)$
        $(\alpha, \beta) = (2, 4)$:$(a, b) = (6, 8)$ → $b = 8$ は不可
        $(\alpha, \beta) = (2, 5)$:$(a, b) = (7, 10)$ → 不可
        $(\alpha, \beta) = (3, 4)$:$(a, b) = (7, 12)$ → 不可

        よって有効な $(a, b)$ は $(3, 2), (4, 3), (5, 4), (6, 5), (5, 6)$ の5通り。

        各場合について、$c$ は $\alpha, \beta$ 以外の1〜6の値を取れる:

        $(a, b) = (3, 2)$:$(\alpha, \beta) = (1, 2)$ なので $c = 3, 4, 5, 6$ で4通り
        $(a, b) = (4, 3)$:$(\alpha, \beta) = (1, 3)$ なので $c = 2, 4, 5, 6$ で4通り
        $(a, b) = (5, 4)$:$(\alpha, \beta) = (1, 4)$ なので $c = 2, 3, 5, 6$ で4通り
        $(a, b) = (6, 5)$:$(\alpha, \beta) = (1, 5)$ なので $c = 2, 3, 4, 6$ で4通り
        $(a, b) = (5, 6)$:$(\alpha, \beta) = (2, 3)$ なので $c = 1, 4, 5, 6$ で4通り

        よって条件を満たす $(a, b, c)$ の組み合わせは $4 \times 5 = 20$ 通り。

        したがって求める確率は $\frac{20}{216} = \frac{5}{54}$

        \textbf{答:(1) $\frac{17}{36}$ (2) $\frac{5}{54}$}}

    \sayA{自分}{もう一度考えてから回答してみてください(ここで熟考ツール使用)}
    \sayB{chatGPT}{以下、答案用紙に記述する体裁で改めて記載します。解説は一切省き、要点だけを示しています。

        \textbf{第5問}

        1個のさいころを3回続けて投げ,出る目を順に $a,b,c$ とする。関数
        $$f(x)=( x^2-ax+b )( x-c )$$
        について,以下に答えよ。

        \textbf{(1)$f(x)=0$ を満たす実数解の個数が1個である確率}

        1. $x^2-ax+b=0$ の判別式を$D=a^2-4b$ とおく。
        \begin{itemize}
            \item $D<0$ のとき,二次式に実数解はなく,$x=c$ の1つだけが実数解 → 解の個数=1。
            \item $D=0$ のとき,二次式の解$r=\frac{a}{2}$が重解。
                  \begin{itemize}
                      \item (i)$c=r$であれば,重解$r$ と $x=c=r$ が同じ値 → 重根はまとめて「解の個数1」
                      \item (ii)$c\neq r$ のとき,二次式の重解$r$と$c$で異なる2つの実数解 → 解の個数2
                  \end{itemize}
        \end{itemize}

        2. 【$D<0$ の場合】
        $a^2-4b<0$ を満たす $(a,b)\in\{1,2,\dots,6\}^2$ の個数を数える。
        \begin{itemize}
            \item $a=1$:$1-4b<0 \iff b\ge1$ → $b=1,\dots,6$ の6通り
            \item $a=2$:$4-4b<0 \iff b\ge2$ → $b=2,\dots,6$ の5通り
            \item $a=3$:$9-4b<0 \iff b\ge3$ → $b=3,\dots,6$ の4通り
            \item $a=4$:$16-4b<0 \iff b\ge5$ → $b=5,6$ の2通り
            \item $a=5$:$25-4b<0 \iff b\ge7$ → なし(0通り)
            \item $a=6$:$36-4b<0 \iff b\ge10$ → なし(0通り)
        \end{itemize}
        よって,$(a,b)$ の通り数 $= 6+5+4+2=17.$
        さらに $c$ は何でもよいので 6 通り。
        $$17\times6 = 102 \text{ 通り}$$

        3. 【$D=0$ の場合】
        $a^2-4b=0\Longrightarrow b=\frac{a^2}{4}, a\in\{1,\dots,6\}.$
        このとき有効なのは
        $$(a,b)=(2,1)\quad(\text{根 }r=1),\quad (a,b)=(4,4)\quad(\text{根 }r=2).$$
        それぞれ,$c=r$ を満たす場合に「解の個数=1」となる。
        \begin{itemize}
            \item $(a,b)=(2,1)$ のとき $r=1$,かつ $c=1$ → 1通り
            \item $(a,b)=(4,4)$ のとき $r=2$,かつ $c=2$ → 1通り
        \end{itemize}
        合計で 2 通り。

        4. 合計の有利事象の数は
        $$102 + 2 = 104 \text{ 通り}.$$
        全体は $6^3=216$ 通りだから,求める確率は
        $$\frac{104}{216} = \frac{13}{27}.$$

        \textbf{(2)$f(x)=0$ を満たす自然数 $x$ の個数が 3 個である確率}

        ※ここで「自然数 $x$ の個数」とは,$x$ の値として自然数を取る解の"個別の値"を数えるものとし,重複はカウントしないものとする。

        1. $f(x)=0$ の解は次の3つ:
        $$x^2-ax+b=0\quad\text{の2解}\,(r_1,r_2) \quad\text{と}\quad x-c=0\,(x=c).$$
        これら3つがすべて自然数かつ互いに異なる必要がある。

        2. 二次式が自然数解をもち,互いに異なる $r_1<r_2\in\mathbb{N}$ を持つとき,
        $$a = r_1 + r_2,\quad b = r_1 r_2.$$
        ただし $r_1,r_2\in\{1,2,3,4,5,6\}$ かつ $a,b\in\{1,\dots,6\}$ を満たす組合せを列挙すると,
        \begin{center}
            \begin{tabular}{c|c|c}
                $(r_1,r_2)$ & $a = r_1 + r_2$ & $b = r_1 r_2$ \\
                \hline
                $(1,2)$     & $3$             & $2$           \\
                $(1,3)$     & $4$             & $3$           \\
                $(1,4)$     & $5$             & $4$           \\
                $(1,5)$     & $6$             & $5$           \\
                $(2,3)$     & $5$             & $6$           \\
            \end{tabular}
        \end{center}
        以上の5通りが $(a,b)$ の候補。

        3. それぞれの $(r_1,r_2)$ に対し,$x=c$ の根 $c$ は $\{1,2,3,4,5,6\}\setminus\{r_1,r_2\}$ の中から選ばなければならない。
        $$\#\{\text{選べる }c\} = 6 - 2 = 4.$$
        したがって,$(a,b)$ ごとに $c$ は 4 通り。
        $$5 \times 4 = 20 \text{ 通り}$$

        4. 全事象数は $6^3=216$ 通り。ゆえに確率は
        $$\frac{20}{216} = \frac{5}{54}.$$

        \textbf{答 (1) $\displaystyle \frac{13}{27}$    (2) $\displaystyle \frac{5}{54}$}

        以上。}
\end{dialogue}
\vskip\baselineskip
chatGPTは前者の問題において,方程式$(x^2-ax+b)(x-c)=0$のうち$x^2-ax+b=0$が重解を持ち$c$に一致するという条件を見落としていた.熟考ツールを使用し再考を促すと,その見落としに気づき修正し正答を得た.

\section{なぜGoogle, OpenAI等は、数学問題を解くことに力を注いでいるのか?}
私は,AIの問題解決能力を数学の問題をものさしにすることで測っているのではないかと考える.
まず,AIを作る際の目標または問題として"現実の問題を解決しうるか"というものがある.
数学の問題というのは,現実の解決したい問題を最小限に切り分けたときの最小限の構成要素であると私は考える.例えば最適化問題(さらに具体的にはナップザック問題など)を思い浮かべると,問題に対して簡易なモデルを想定し数学の力を使って解いている.
現実の問題ではそのまま数学をつかえるとは限らないが,課題を分割していけば結局数学の問題を解くことと同じになるのである.(306字)

\section{数学的な推論能力の進化を体験して、気づきや将来の協働・役割分担はどうなると思うか?}
AIは,膨大なデータによる蓄積や計算と論理的処理を瞬時に処理できるが,現在のところ人の手で調整が必要な場面もある.
これは言い換えると,「人が」問題解決のためにその問題に合うようにチューニングし,AIを「使う」のである.\vskip.5 \baselineskip
人は,直感や創造的な発想,問題設定などアイデアを生み出すことに長けている.AIのような膨大な計算やデータを扱うことは得意ではない.\vskip.5\baselineskip
以上から,複雑な計算やパターン認識,データを扱う分野はAIがにない,それを用いて人が社会的文脈に還元することで人とAIが協力していく将来を私は想像している.(294字)


\end{document}
