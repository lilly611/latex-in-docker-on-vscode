% This document is based on http://math.shinshu-u.ac.jp/~hanaki/beamer/beamer.html
\documentclass[dvipdfmx,cjk]{beamer}
%\documentclass[dvipdfm,cjk]{beamer} % オプションは環境や利用するプログラムによって変える
%\documentclass[dvips,cjk]{beamer}
\usepackage{amsmath}
\usepackage{amssymb}
\usepackage{amsfonts}
\usepackage{latexsym}


% しおり(PDF にしたときの目次)の文字化け防止
\AtBeginDvi{\special{pdf:tounicode 90ms-RKSJ-UCS2}}
%\AtBeginDvi{\special{pdf:tounicode EUC-UCS2}}

% 右下のアイコンを消す
\setbeamertemplate{navigation symbols}{}

% テーマ
\usetheme{CambridgeUS}
%\usetheme{Boadilla}           %% Beamer のディレクトリの中の
%\usetheme{Madrid}             %% beamerthemeCambridgeUS.sty を指定
%\usetheme{Antibes}            %% 色々と試してみるといいだろう
%\usetheme{Montpellier}        %% サンプルが beamer\doc に色々とある。
%\usetheme{Berkeley}
%\usetheme{Goettingen}
%\usetheme{Singapore}
%\usetheme{Szeged}

\usecolortheme{rose}          %% colortheme を選ぶと色使いが変わる
%\usecolortheme{albatross}

%\useoutertheme{shadow}                 %% 箱に影をつける
\usefonttheme{professionalfonts}       %% 数式の文字を通常の LaTeX と同じにする

%\setbeamercovered{transparent}         %% 消えている文字をうっすらと表示する
%\setbeamertemplate{theorems}[numbered]  %% 定理に番号をつける
\newtheorem{thm}{Theorem}[section]
\newtheorem{proposition}[thm]{Proposition}
\theoremstyle{example}
\newtheorem{exam}[thm]{Example}
\newtheorem{remark}[thm]{Remark}
\newtheorem{question}[thm]{Question}
\newtheorem{prob}[thm]{Problem}
\newtheorem{rev}[thm]{Review}


% メタ情報
\begin{document}
\title[]{カーネル主成分分析}
\author[]{照屋 佑喜仁}
\institute[]{}
\date{\today}

% タイトルスライド
\begin{frame}
    \titlepage
\end{frame}

% 目次(\section 名が自動で挿入される)
\begin{frame}
    \tableofcontents
\end{frame}

% セクション名(サンプルでは左上のスペースに表示される)
\section{高次元空間への写像}
\begin{frame}
    \frametitle{高次元空間への写像} % スライドのタイトル
    \begin{itemize}
        \item 前回の最後で,カーネル法に基づく主成分分析の基本的な考え方が述べられた.
        \item 高次元空間に散らばるデータの非線形構造を捉える一つのアプローチがカーネル主成分分析であった.
        \item 中心化したデータの分散共分散行列$S$の固有値問題を,データ間の内積で定義される新たな行列$K$の固有値問題に置き換えて主成分を求めることができそう
        \item この項で,固有値問題の置き換えを,高次元空間へと写像したデータに対して適応する.
    \end{itemize}

\end{frame}

\begin{frame}
    \frametitle{高次元空間への写像}
    入力空間上の$n$個の$p$次元観測データ$\boldsymbol{x}_1,\boldsymbol{x}_2,\cdots ,\boldsymbol{x}_n$を\textbf{特徴空間}とよばれる高次元空間へ写像し,特徴空間上で主成分分析を実行する.
    \vskip.5\baselineskip
    一般に,入力空間上の$p$次元データ$\boldsymbol{x}=(x_1,x_2,\cdots,x_p)^T$を,より高次の$r$次元空間へと変換する非線形空間を次のようにベクトルで表記する.
    \begin{align*}
        \boldsymbol{\Phi}(\boldsymbol{x})\equiv(\phi_1(\boldsymbol{x}),\phi_2(\boldsymbol{x}),\cdots,\phi_r(\boldsymbol{x}))^T
    \end{align*}
    このベクトルの各成分$\phi_j(\boldsymbol{x})$は$p$変数実数値関数とする.\vskip.5\baselineskip
    これのベクトルによって,観測された$n$個の$p$次元データを$r$次元特徴空間上へと写像したデータを
    \begin{align*}
        \boldsymbol{\Phi}(\boldsymbol{x}_1),\boldsymbol{\Phi}(\boldsymbol{x}_2),\cdots,\boldsymbol{\Phi}(\boldsymbol{x}_n)
    \end{align*}
    とする.この特徴空間上の$n$個の$r$次元データに基づいて主成分分析を実行する.
\end{frame}

\begin{frame}
    \frametitle{高次元空間への写像}
    まず,特徴空間上に散らばる$r$次元データの重心に原点を移す.(つまり中心化する.)\vskip.5\baselineskip
    標本平均ベクトルは
    \begin{align*}
        \overline{\boldsymbol{\Phi}}=\frac{1}{\;n\;}\sum_{i=1}^{n}\boldsymbol{\Phi}(\boldsymbol{x}_i)
    \end{align*}
    となり,次のようにデータを中心化を行う.
    \begin{align*}
        \boldsymbol{\Phi}_c(\boldsymbol{x}_i)=\boldsymbol{\Phi}(\boldsymbol{x}_i)-\overline{\boldsymbol{\Phi}},\quad i=1,2,\cdots,n
    \end{align*}
\end{frame}

\begin{frame}
    \frametitle{高次元空間への写像}
    中心化したデータの標本平均ベクトルは
    \begin{align*}
        \frac{1}{\;n\;}\sum_{i=1}^{n}\boldsymbol{\Phi}_c(\boldsymbol{x}_i)=\frac{1}{\;n\;}\sum_{i=1}^{n}\left\{\boldsymbol{\Phi}(\boldsymbol{x}_i)-\overline{\boldsymbol{\Phi}}\right\}=\boldsymbol{0}
    \end{align*}
    (中心化したため当たり前)\vskip.5\baselineskip
    次に,中心化したデータを行ベクトルとする$n\times r$行列を
    \begin{align*}
        Z_c=\left(\boldsymbol{\Phi}_c(\boldsymbol{x}_1),\boldsymbol{\Phi}_c(\boldsymbol{x}_2),\cdots,\boldsymbol{\Phi}_c(\boldsymbol{x}_n)\right)^T
    \end{align*}
    とすると,標本分散共分散行列は
    \begin{align*}
        S_c=\frac{1}{\;n\;}\sum_{i=1}^{n}\boldsymbol{\Phi}_c(\boldsymbol{x}_i)\boldsymbol{\Phi}_c(\boldsymbol{x}_i)^T=\frac{1}{\;n\;}Z_c^TZ_c
    \end{align*}
\end{frame}

\begin{frame}
    \frametitle{高次元空間への写像}
    前回行ったように,分散共分散行列$S_c$の固有値問題を,特徴空間上のデータ間の内積に基づく行列の固有値問題へと置き換える.\vskip.5\baselineskip
    一般に,行列$S_c$の第$\alpha$番目の固有値$\lambda_\alpha^F$と対応する正規化済みの固有ベクトル$\boldsymbol{w}_\alpha^F$の関係式
    \begin{align*}
        S_c\boldsymbol{w}_\alpha^F=\lambda_\alpha^F\boldsymbol{w}_\alpha^F\tag{1}
    \end{align*}
    から,固有ベクトルは,中心化した特徴空間上のデータを基底として次のように表すことができるのであった.(前回)
\end{frame}

\begin{frame}
    \frametitle{高次元空間への写像}
    \begin{rev}
        \begin{align*}
            S_c\boldsymbol{w}_\alpha^F=\lambda_\alpha^F\boldsymbol{w}_\alpha^F
        \end{align*}
        の左辺は次のように書き表せる
        \begin{align*}
            S_c\boldsymbol{w}_\alpha^F & =\frac{1}{\;n\;}\sum_{i=1}^{n}\boldsymbol{\Phi}_c(\boldsymbol{x}_i)\boldsymbol{\Phi}_c(\boldsymbol{x}_i)^T\boldsymbol{w}_\alpha^F                                                                                                         \\
                                       & =\frac{1}{\;n\;}\sum_{i=1}^{n}\left\{\boldsymbol{\Phi}_c(\boldsymbol{x}_i)^T\boldsymbol{w}_\alpha^F\right\}\boldsymbol{\Phi}_c(\boldsymbol{x}_i)\quad\because\text{$\boldsymbol{\Phi}_c(\boldsymbol{x}_i)^T\boldsymbol{w}_\alpha^F$はスカラー}
        \end{align*}
    \end{rev}
\end{frame}

\begin{frame}
    \frametitle{高次元空間への写像}
    改めて固有ベクトルは次で表せる.
    \begin{align*}
        \boldsymbol{w}_\alpha^F=Z_c^T\boldsymbol{c}_\alpha^F,\quad Z_c^T=\left(\boldsymbol{\Phi}_c(\boldsymbol{x}_1),\boldsymbol{\Phi}_c(\boldsymbol{x}_2),\cdots,\boldsymbol{\Phi}_c(\boldsymbol{x}_n),\right)\tag{3}
    \end{align*}
    ただし,$\boldsymbol{c}_\alpha^T$は$n$次元係数ベクトルとする.これを(1)に代入すると
    \begin{align*}
        \frac{1}{\;n\;}Z_c^TZ_cZ_c^T\boldsymbol{c}_\alpha^F=\lambda_\alpha^FZ_c^T\boldsymbol{c}_\alpha^F
    \end{align*}
    左から$Z_c$を掛けて分母を払うと
    \begin{align*}
        Z_cZ_c^TZ_cZ_c^T\boldsymbol{c}_\alpha^F=n\lambda_\alpha^FZ_cZ_c^T\boldsymbol{c}_\alpha^F\tag{2}
    \end{align*}
    を得る.$Z_cZ_c^T$は次のように特徴空間上の中心化したデータ間の内積を成分とする$n$次元対称行列である.
\end{frame}
\begin{frame}
    \frametitle{高次元空間への写像}
    \begin{align*}
        K_c & \equiv Z_cZ_c^T                                                                                                                                                                                                                                      \\
            & =\begin{pmatrix}
                   \boldsymbol{\Phi}_c(\boldsymbol{x}_1)^T\boldsymbol{\Phi}_c(\boldsymbol{x}_1) & \boldsymbol{\Phi}_c(\boldsymbol{x}_1)^T\boldsymbol{\Phi}_c(\boldsymbol{x}_2) & \cdots & \boldsymbol{\Phi}_c(\boldsymbol{x}_1)^T\boldsymbol{\Phi}_c(\boldsymbol{x}_n) \\
                   \boldsymbol{\Phi}_c(\boldsymbol{x}_2)^T\boldsymbol{\Phi}_c(\boldsymbol{x}_1) & \boldsymbol{\Phi}_c(\boldsymbol{x}_2)^T\boldsymbol{\Phi}_c(\boldsymbol{x}_2) & \cdots & \boldsymbol{\Phi}_c(\boldsymbol{x}_2)^T\boldsymbol{\Phi}_c(\boldsymbol{x}_n) \\
                   \vdots                                                                       & \vdots                                                                       & \ddots & \vdots                                                                       \\
                   \boldsymbol{\Phi}_c(\boldsymbol{x}_n)^T\boldsymbol{\Phi}_c(\boldsymbol{x}_1) & \boldsymbol{\Phi}_c(\boldsymbol{x}_n)^T\boldsymbol{\Phi}_c(\boldsymbol{x}_2) & \dots  & \boldsymbol{\Phi}_c(\boldsymbol{x}_n)^T\boldsymbol{\Phi}_c(\boldsymbol{x}_n)
               \end{pmatrix}
    \end{align*}
    よって(2)は次のように表される.
    \begin{align*}
        K_c^2\boldsymbol{c}_\alpha^F=n\lambda_\alpha^F\boldsymbol{c}_\alpha^F
    \end{align*}
\end{frame}
\begin{frame}
    \frametitle{高次元空間への写像}
    したがって,第$\alpha$番目の大きさの固有値$\lambda_\alpha^F(\neq0)$と対応する固有ベクトル$\boldsymbol{c}_\alpha^F$を求める問題は,次の$n$次対称行列$K_c$の固有値問題へと帰着される.
    \begin{align*}
        K_c\boldsymbol{c}_\alpha^F=n\lambda_\alpha^F\boldsymbol{c}_\alpha^F
    \end{align*}
    求めた固有ベクトル$\boldsymbol{c}_\alpha^F$を(3)へ代入すると
    \begin{align*}
        \boldsymbol{w}_\alpha^F=Z_c^T\boldsymbol{c}_\alpha^F=\sum_{i=1}^{n}c_{i\alpha}^F\boldsymbol{\Phi}_c(\boldsymbol{x}_i)
    \end{align*}
    を得る.ただし行列$K_c$の固有ベクトルは次の条件を満たす.
    \begin{align*}
        1 & ={\boldsymbol{w}_\alpha^F}^T\boldsymbol{w}_\alpha^F\quad \because\text{$\boldsymbol{w}_\alpha^F$は正規化済}                                                                                                                                             \\
          & =(Z_c^T\boldsymbol{c}_\alpha^F)^T(Z_c^T\boldsymbol{c}_\alpha^F)={\boldsymbol{c}_\alpha^F}^TZ_cZ_c^T\boldsymbol{c}_\alpha^F={\boldsymbol{c}_\alpha^F}^TK\boldsymbol{c}_\alpha^F=n\lambda_\alpha^F{\boldsymbol{c}_\alpha^F}^T\boldsymbol{c}_\alpha^F
    \end{align*}
\end{frame}
\begin{frame}
    \frametitle{高次元空間への写像}
    以上より,特徴空間上の中心化されたデータに基づいて構成した第$\alpha$番目の主成分は
    \begin{align*}
        y_\alpha^F={\boldsymbol{w}_\alpha^F}^T\boldsymbol{\Phi}_c(\boldsymbol{x})=\sum_{i=1}^{n}c_{i\alpha}^F{\boldsymbol{\Phi}_c(\boldsymbol{x})}^T\boldsymbol{\Phi}_c(\boldsymbol{x})
    \end{align*}
    で与えられる.\vskip.5\baselineskip
    入力空間の次元が高いと極めて高次元の特徴空間へとデータを写像する必要があり,これによってデータ間の内積${\boldsymbol{\Phi}_c(\boldsymbol{x})}^T\boldsymbol{\Phi}_c(\boldsymbol{x})$の計算が困難となる.\\
    このような状況に対処するために有用なのがカーネル法である.
\end{frame}
\section{カーネル法}
\begin{frame}
    \frametitle{カーネル法}
    先程述べたように,高次元ベクトル間の内積の計算の問題がある.これを次のようにカーネル関数で置き換えることで克服する.
    \begin{align*}
        K_c(\boldsymbol{x}_j,\boldsymbol{x}_k)={\boldsymbol{\Phi}_c(\boldsymbol{x}_j)}^T\boldsymbol{\Phi}_c(\boldsymbol{x}_k)
    \end{align*}
    右辺の値を,計算可能な左辺のカーネル関数の値として求めているといえる.例えば,代表的なカーネル関数であるガウスカーネルを用いると
    \begin{align*}
        K_c(\boldsymbol{x}_j,\boldsymbol{x}_k)=\exp\left\{-\frac{{||\boldsymbol{x}_j-\boldsymbol{x}_k||}^2}{2\sigma^2}\right\}={\boldsymbol{\Phi}_c(\boldsymbol{x}_j)}^T\boldsymbol{\Phi}_c(\boldsymbol{x}_k)\tag{4}
    \end{align*}
    と表される.右辺の内積の計算の計算は極度の高次元化により困難であっても,左辺の計算は入力空間上の$p$次元データの内積に基づく計算であるから用意に求まることがわかる.
\end{frame}
\begin{frame}
    \frametitle{カーネル法}
    $K_c$は,計算可能なカーネル関数を成分とする行列で表すことができる.
    \begin{align*}
        K_c & \equiv Z_cZ_c^T                                                                                                                    \\
            & =\begin{pmatrix}
                   K_c(\boldsymbol{x}_1,\boldsymbol{x}_1) & K_c(\boldsymbol{x}_1,\boldsymbol{x}_2) & \dots  & K_c(\boldsymbol{x}_1,\boldsymbol{x}_n) \\
                   K_c(\boldsymbol{x}_2,\boldsymbol{x}_1) & K_c(\boldsymbol{x}_2,\boldsymbol{x}_2) & \dots  & K_c(\boldsymbol{x}_2,\boldsymbol{x}_n) \\
                   \vdots                                 & \vdots                                 & \ddots & \vdots                                 \\
                   K_c(\boldsymbol{x}_n,\boldsymbol{x}_1) & K_c(\boldsymbol{x}_n,\boldsymbol{x}_2) & \dots  & K_c(\boldsymbol{x}_n,\boldsymbol{x}_n)
               \end{pmatrix}
    \end{align*}
    同様に,第$\alpha$番目の主成分も
    \begin{align*}
        y_\alpha^F={\boldsymbol{w}_\alpha^F}^T\boldsymbol{\Phi}_c(\boldsymbol{x})=\sum_{i=1}^{n}c_{i\alpha}^F{\boldsymbol{\Phi}_c(\boldsymbol{x})}^T\boldsymbol{\Phi}_c(\boldsymbol{x})=\sum_{i=1}^{n}c_{i\alpha}^FK_c(\boldsymbol{x}_i,\boldsymbol{x})
    \end{align*}
    と表せる.
\end{frame}
\begin{frame}
    \frametitle{カーネル法}
    データの中心化の過程で必要となる標本平均ベクトル$\overline{\boldsymbol{\Phi}}$をどのように計算するかという問題がある.(?)この問題を避けるためには,高次元特徴空間へ変換した$r$次元データ$\boldsymbol{\Phi}(\boldsymbol{x}_1),\boldsymbol{\Phi}(\boldsymbol{x}_2),\dots,\boldsymbol{\Phi}(\boldsymbol{x}_n)$の内積をカーネル関数で置き換えた式
    \begin{align*}
        K(\boldsymbol{x}_j,\boldsymbol{x}_k)={\boldsymbol{\Phi}(\boldsymbol{x}_j)}\boldsymbol{\Phi}(\boldsymbol{x}_k)\tag{5}
    \end{align*}
    を用いて(4)の中心化したデータに対するカーネル関数を表す必要がある.
\end{frame}
\begin{frame}
    \frametitle{カーネル法}
    \begin{align*}
        K_c(\boldsymbol{x}_j,\boldsymbol{x}_k) & ={\boldsymbol{\Phi}_c(\boldsymbol{x}_j)}^T\boldsymbol{\Phi}_c(\boldsymbol{x}_k)                                                                                                                                                       \\
                                               & =\left(\boldsymbol{\Phi}(\boldsymbol{x}_j)-\frac{1}{\;n\;}\sum_{r=1}^{n}\boldsymbol{\Phi}(\boldsymbol{x}_r)\right)^T\left(\boldsymbol{\Phi}(\boldsymbol{x}_k)-\frac{1}{\;n\;}\sum_{s=1}^{n}\boldsymbol{\Phi}(\boldsymbol{x}_s)\right) \\
                                               & =K(\boldsymbol{x}_j,\boldsymbol{x}_k)-\frac{1}{\;n\;}\sum_{r=1}^{n}K(\boldsymbol{x}_r,\boldsymbol{x}_k)-\frac{1}{\;n\;}\sum_{s=1}^{n}K(\boldsymbol{x}_j,\boldsymbol{x}_s)                                                             \\&\quad+\frac{1}{\;n^2\;}\sum_{r=1}^{n}\sum_{s=1}^{n}K(\boldsymbol{x}_r,\boldsymbol{x}_s)
    \end{align*}
\end{frame}
\begin{frame}
    \frametitle{カーネル法}
    さらに,中心化したデータに基づく,内積で定義されたカーネル関数を成分とする行列$K_c$を(5)で定義されるカーネル関数を成分とする行列$K$で置き換える.このため,特徴空間上の$r$次元データを行ベクトルとする$n\times r$データ行列を次のように定義する.
    \begin{align*}
        Z=\left(\boldsymbol{\Phi}(\boldsymbol{x}_1),\boldsymbol{\Phi}(\boldsymbol{x}_2),\dots,\boldsymbol{\Phi}(\boldsymbol{x}_n)\right)^T
    \end{align*}
    このとき標本平均ベクトルは
    \begin{align*}
        \overline{\boldsymbol{\Phi}}=\frac{1}{\;n\;}\sum_{i=1}^{n}\boldsymbol{\Phi}(\boldsymbol{x}_i)=\frac{1}{\;n\;}Z^T\boldsymbol{1}_n
    \end{align*}
    と表せる.ただし,$\boldsymbol{1}_n$はその成分がすべて$1$の$n$次元ベクトルとする.また,中心化したデータを行ベクトルとするデータ行列$Z_c$は
\end{frame}
\begin{frame}
    \frametitle{カーネル法}
    \begin{align*}
        Z_c=
        \begin{pmatrix}
            {\boldsymbol{\Phi}_c(\boldsymbol{x}_1)}^T \\
            {\boldsymbol{\Phi}_c(\boldsymbol{x}_2)}^T \\
            \vdots                                    \\
            {\boldsymbol{\Phi}_c(\boldsymbol{x}_n)}^T
        \end{pmatrix}
        =
        \begin{pmatrix}
            {\boldsymbol{\Phi}(\boldsymbol{x}_1)}^T \\
            {\boldsymbol{\Phi}(\boldsymbol{x}_2)}^T \\
            \vdots                                  \\
            {\boldsymbol{\Phi}(\boldsymbol{x}_n)}^T
        \end{pmatrix}
        -
        \begin{pmatrix}
            1 \\1\\ \vdots \\1
        \end{pmatrix}\overline{\boldsymbol{\Phi}}^T=
        Z-\frac{1}{\;n\;}\boldsymbol{1}_n\boldsymbol{1}_n^TZ
    \end{align*}
    と表すことができる.\vskip.5\baselineskip
    この結果から
\end{frame}
\begin{frame}
    \frametitle{カーネル法}
    \begin{align*}
        K_c=Z_cZ_c^T & =\left(Z-\frac{1}{\;n\;}\boldsymbol{1}_n\boldsymbol{1}_c^TZ\right)\left(Z-\frac{1}{\;n\;}\boldsymbol{1}_n\boldsymbol{1}_c^TZ\right)^T       \\
                     & =\left(I_n-\frac{1}{\;n\;}\boldsymbol{1}_n\boldsymbol{1}_n^T\right)ZZ^T\left(I_n-\frac{1}{\;n\;}\boldsymbol{1}_n\boldsymbol{1}_n^T\right)^T
    \end{align*}
    で与えられる.ただし$I_n$は$n$次単位行列である.したがって,中心化したデータに基づく内積をカーネル関数で置き換えた$K_c=Z_cZ_c^T$は,特徴空間へ変換したデータに基づく内積を,カーネル関数で置き換えた$n$次対称行列$K=ZZ^T$によって
    \begin{align*}
        K_c=\left(I_n-\frac{1}{\;n\;}\boldsymbol{1}_n\boldsymbol{1}_n^T\right)K\left(I_n-\frac{1}{\;n\;}\boldsymbol{1}_n\boldsymbol{1}_n^T\right)^T
    \end{align*}
    と表すことができる.
\end{frame}
\section{非線形主成分分析のプロセス}
\begin{frame}
    \frametitle{非線形主成分分析のプロセス}
    $n$個の$p$次元データ$\boldsymbol{x}_1,\boldsymbol{x}_2,\dots,\boldsymbol{x}_n$が観測されたとする.
    \begin{itemize}
        \item 入力空間上の観測データを非線形写像$\boldsymbol{\Phi}(\boldsymbol{x})=\left(\phi_1(\boldsymbol{x}),\phi_2(\boldsymbol{x}),\dots,\phi_r(\boldsymbol{x})\right)^T$によって高次元特徴空間へと変換した$n$個の$r$次元データを
              \begin{align*}
                  \boldsymbol{\Phi}(\boldsymbol{x}_1),\boldsymbol{\Phi}(\boldsymbol{x}_2),\cdots,\boldsymbol{\Phi}(\boldsymbol{x}_n)
              \end{align*}とする.
        \item 特徴空間上の$r$次元データ間の内積をカーネル関数で置き換える.
              \begin{align*}
                  K(\boldsymbol{x}_j,\boldsymbol{x}_k)={\boldsymbol{\Phi}(\boldsymbol{x}_j)}^T\boldsymbol{\Phi}(\boldsymbol{x}_k)
              \end{align*}
    \end{itemize}
\end{frame}
\begin{frame}
    \frametitle{非線形主成分分析のプロセス}
    \begin{itemize}
        \item カーネル関数$K(\boldsymbol{x}_j,\boldsymbol{x}_k)$を第$(j,k)$成分とする$n$次カーネル行列を$K$として,中心化したデータ間の内積を成分とする$n$次カーネル行列$K_c$を求める.
              \begin{align*}
                  K_c=\left(I_n-\frac{1}{\;n\;}\boldsymbol{1}_n\boldsymbol{1}_c^T\right)K\left(I_n-\frac{1}{\;n\;}\boldsymbol{1}_n\boldsymbol{1}_c^T\right)^T
              \end{align*}
        \item $K_c$に対して,条件$n\lambda^F{\boldsymbol{c}^F}^T\boldsymbol{c}^F$のもとで次の固有値問題の解を求める.
              \begin{align*}
                  K_c\boldsymbol{c}^F=n\lambda^F\boldsymbol{c}^F
              \end{align*}
              大きさの順に並べた固有値の中で$\alpha$番目の$0$でない固有値に対応するベクトルを$\boldsymbol{c}_\alpha^F=(c_{1\alpha}^F,c_{2\alpha}^F,\cdots,c_{n\alpha}^F)^T$とすると,第$\alpha$番目の主成分は次で与えられる.
              \begin{align*}
                  y_\alpha=\sum_{i=1}^{n}c_{i\alpha}^FK_c(\boldsymbol{x}_i,\boldsymbol{x})
              \end{align*}

    \end{itemize}

\end{frame}
\begin{frame}
    \frametitle{非線形主成分分析のプロセス}
    ただし$K_c$は中心化したデータ間の内積をカーネル関数で置き換えたものである.
    \begin{align*}
        K_c(\boldsymbol{x}_j,\boldsymbol{x}_k) & ={\boldsymbol{\Phi}_c(\boldsymbol{x}_j)}^T\boldsymbol{\Phi}_c(\boldsymbol{x}_k)                                                                                                                                                       \\
                                               & =\left(\boldsymbol{\Phi}(\boldsymbol{x}_j)-\frac{1}{\;n\;}\sum_{r=1}^{n}\boldsymbol{\Phi}(\boldsymbol{x}_r)\right)^T\left(\boldsymbol{\Phi}(\boldsymbol{x}_k)-\frac{1}{\;n\;}\sum_{s=1}^{n}\boldsymbol{\Phi}(\boldsymbol{x}_s)\right) \\
                                               & =K(\boldsymbol{x}_j,\boldsymbol{x}_k)-\frac{1}{\;n\;}\sum_{r=1}^{n}K(\boldsymbol{x}_r,\boldsymbol{x}_k)-\frac{1}{\;n\;}\sum_{s=1}^{n}K(\boldsymbol{x}_j,\boldsymbol{x}_s)                                                             \\&\quad+\frac{1}{\;n^2\;}\sum_{r=1}^{n}\sum_{s=1}^{n}K(\boldsymbol{x}_r,\boldsymbol{x}_s)
    \end{align*}
\end{frame}
\end{document}
