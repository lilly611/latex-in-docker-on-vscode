\documentclass[a4j,uplatex]{jsarticle}
\usepackage{amsmath}
\usepackage{amsthm}
\usepackage{amssymb}
\usepackage{amsfonts}
\theoremstyle{definition}
\newtheorem{dfn}{Definition}[section]
\newtheorem*{df*}{Definition}
\newtheorem{prop}[dfn]{Proposition}
\newtheorem{lem}[dfn]{Lemma}
\newtheorem{thm}[dfn]{Theorem}
\newtheorem{cor}[dfn]{Corollary}
\newtheorem{rem}[dfn]{Remark}
\newtheorem{fact}[dfn]{Fact}

\title{院試過去問でわからなかったところの教科書まとめ}
\author{照屋佑喜仁\thanks{参考:斎藤微積}}

\begin{document}
\maketitle
院試に向けて自分用まとめ 順番は適当
\tableofcontents

\section{積分}
\subsection{テクニック}
\begin{rem}[$\cos^2x$などの積分]
    三角関数の累乗の積分は2倍角などで次数を下げるとうまくいくことがある.
\end{rem}
\subsection{面積・長さ}
\subsubsection{極座標}
\begin{df*}[曲領域の面積]
    極座標で$r=f(\theta)$なる曲線と2直線$\theta=a,\theta=b(a<b)$とで囲まれる領域を\emph{曲領域}という.\\
    その面積$S$を
    \begin{align*}
        S=\frac{1}{2}\int_{a}^{b}r^2d\theta=\frac{1}{2}\int_{a}^{b}f(\theta)^2d\theta
    \end{align*}
\end{df*}

\subsubsection{長さ}
\begin{df*}
    閉曲線$C$
    \begin{align*}
        x=x(t),\: y=y(t) \quad \alpha \leq t \leq \beta
    \end{align*}
    の長さ$\ell(C)$は次で定める
    \begin{align*}
        \ell(C)=\int_{\alpha}^{\beta}\sqrt{\left(\frac{dx(t)}{dt}\right)^2+\left(\frac{dy(t)}{dt}\right)^2}dt
    \end{align*}
    特に,$t=x\: (a\leq x\leq b)$,\ $y=f(x)$で表される曲線の長さは
    \begin{align*}
        \int_{a}^{b}\sqrt{1+\left\{f'(x)\right\}^2}dt
    \end{align*}
\end{df*}
\end{document}
